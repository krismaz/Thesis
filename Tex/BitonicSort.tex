\FloatBarrier
\section{Bitonic Sort} 
\label{sec:BitonicSort}

Bitonic Sort is one of the earliest sorting networks, and was presented in~\citeA{SNApplications}. The algorithm is based on the concept of bitonic sequences, which is a sequence constructed as the juxtaposition of an ascending and a descending sequence, and the operation of the algorithm is based on constructing and merging bitonic sequences.

The algorithm suffers from a running time of $\Theta(n \log^2 n)$, but it works well in practice due to having small constants, and being suitable for compile-time optimisation.

\subsection{Bitonic Merging}

The one basic idea behind Bitonic Sort, is that any bitonic sequence can be sorted with relative ease by a sorting network.

The main part of this sorting consist of a bitonic split\footnote{
The \emph{bitonic split} is also referred to as the \emph{bitonic merge} in most of this thesis, despite not actually merging sequences. The \emph{merge} misnomer originates in the structure of the algorithm, where the \emph{bitonic split} takes the role of a merge in a classic mergesort. The \emph{bitonic merge} merge may also be seen as merging an ascending and a descending sequence into a single sorted sequence.
}, which given a bitonic sequence $A$ of length $N$ creates

\begin{equation}
\begin{split}
&HI = \{\max(a_1, a_{n/2+1}), \max(a_2, a_{n/2+2}), \max(a_3, a_{n/2+3}) \dots \max(a_{n/2}, a_n)\}\\
\\
&LO = \{\min(a_1, a_{n/2+1}), \min(a_2, a_{n/2+2}), \min(a_3, a_{n/2+3}) \dots \min(a_{n/2}, a_n)\}
\end{split}
\end{equation}

\noindent
where $HI$ and $LO$ are bitonic sequences of length $N/2$, and any element in $HI$ is bigger than all elements $LO$. These properties of $HI$ and $LO$ are determined in~\citeA{SNApplications}. 

Since the bitonic split moves the elements to a low and a high half, and both halves are still bitonic sequences, we can sort a bitonic sequence by recursively applying bitonic split. The recursive operation is described in Algorithm~\ref{BitonicMerge}

\begin{algorithm}
\caption{Bitonic Merge}\label{BitonicMerge}
\begin{algorithmic}[1]
	\Statex $A$: Bitonic array input of \texttt{Compare-Exchange} compatible elements
	\Statex $n$: Size of $A$
\Procedure{BitonicMerge}{$A, n$}
\If {$n>1$}
	\For {$i = 0 \dots n/2-1$}
		\State {$\mathtt{Compare\mbox{-}Exchange}(A, i, i+n/2)$}
	\EndFor
	\State{$\mathtt{BitonicMerge}(A[0 \dots n/2-1], n/2)$}
	\State{$\mathtt{BitonicMerge}(A[n/2 \dots n-1], n/2)$}
\EndIf
\EndProcedure
\end{algorithmic}
\end{algorithm}

\subsection{The Main Algorithm}

Given the bitonic merge operation, we can construct a sorting algorithm by exploiting the fact that the concatenation of an ascending and a descending sequence will be a bitonic sequence.

\begin{thm}
If we can sort bitonic sequences, then we can sort any sequence of length $2^k$
\end{thm}

\begin{proof}
Simple induction proof:

\textbf{Base case - $2^0$:} Any sequence of length $2^0 = 1$ is already sorted.

\textbf{Inductive case - $2^k$:} Let $A$ be the first $2^{k-1}$ elements, and $B$ the last $2^{k-1}$ elements. Sort $A$ and $B$, reverse $A$, let $C = A+B$. $C$ is now a bitonic sequence containing the original elements, and since we can sort bitonic sequences using Algorithm \ref{BitonicMerge}, we can sort the original elements.
\end{proof}


This gives us a fairly simple way of sorting using the bitonic merging procedure, as shown in Algorithm~\ref{BitonicSort}.

\begin{algorithm}
\caption{Bitonic Sort}\label{BitonicSort}
\begin{algorithmic}[1]
	\Statex $A$: Array input of \texttt{Compare-Exchange} compatible elements
	\Statex $n$: Size of $A$
\Procedure{BitonicSort}{$A, n$}
\If {$n>1$}
	\State{$\mathtt{BitonicSort}(A[0 \dots n/2-1], n/2)$}
	\State{$\mathtt{BitonicSort}(A[n/2 \dots n-1], n/2)$}
	\State{$\mathtt{Reverse}(A[0 \dots n/2-1])$}
	\State{$\mathtt{BitonicMerge}(A, n)$}
\EndIf
\EndProcedure
\end{algorithmic}
\end{algorithm}

The $\Theta(n \log^2 n)$ total amount of \texttt{Compare-Exchange} operations performed by the algorithm follows from the Master Theorem \textcolor{red}{and exercise 4.6-2} of~\citeB{Cormen}.




