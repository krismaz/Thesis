\FloatBarrier
\section{Odd-Even Mergesort} 
\label{sec:OddEvenMergesort}

Odd-Even Mergesort is the sister algorithm of Bitonic Sort, and also comes from~\citeA{SNApplications}. The idea of requiring bitonic sequences for merging in  Bitonic Sort seems unintuitive, and Odd-Even Mergesort does away with that requirement, at the expense of slightly more complicated merging operation.
Like Bitonic Sort, Odd-Even Mergesort performs $\Theta(n \log^2 n)$ comparisons, but the constant factors involved are slightly lower.

\subsection{Odd-Even Merging}

Given two sorted sequences, it is possible to merge them by dividing them into their odd and even parts, and then combining them into an interleaved sequence.

The intuition of this merging scheme follows from the following observation:

\noindent
Let $A$ and $B$ be sorted sequences of length $N$.

\noindent
Let $C$ be the sorted merge of $A$ and $B$.

$C$ might look something like this:

\[
C = {a_0, b_0, b_1, b_2, a_1, a_2, b_3 \dots}
\]

or

\[
C = {b_0, b_1, a_0, b_2, a_1, b_3, b_4 \dots}
\]

Now, imagine a grouping of $C$ as follows; 

\[
C = {a_0,(b_0, b_1), (b_2, a_1), (a_2, b_3) \dots}
\]

or

\[
C = {b_0, (b_1, a_0), (b_2, a_1), (b_3, b_4) \dots}
\]

What we see here, is that every pair of parenthesized numbers is made up of one odd-indexed number and one even-indexed. In fact~\citeA{SNApplications} shows that every merged sequence of two sorted sequences will follow this rule, and we can use this to construct $C$ in the following way.

Divide $A$ and $B$ into odd and even indexes, and merge them recursively. Pick $c_0 = even_0$, pick $c_1 = \min(odd_0, even_1)$, $c_2 = \max(odd_0, even_1)$, $c_3 = \min(odd_1, even_2)$, $c_4 = \max(odd_1, even_2)$ \dots and so on. In case there is only one element in each subsequence, just perform a single \texttt{Compare-Exchange}. This gives us the following algorithm, shown as pseudo-code in Algorithm~\ref{OddEvenMerge}.

\begin{algorithm}
\caption{Odd-Even Merge}\label{OddEvenMerge}
\begin{algorithmic}[1]
	\Statex $A$: Array input of \texttt{Compare-Exchange} compatible elements
	\Statex $B$: Array input of \texttt{Compare-Exchange} compatible elements
	\Statex $n$: Size of $A$ and $B$
\Procedure{OddEvenMerge}{$A, B, n$}
\If {$n=1$}
	\State $C \gets [{A[0], B[0]}]$
	\State $\mathtt{Compare-Exchange}(C, 0, 1)$
\Else
	\State {$odd \gets \mathtt{OddEvenMerge}(\mathtt{odd}(A),\mathtt{odd}(B), n/2)$}
	\State {$even \gets \mathtt{OddEvenMerge}(\mathtt{even}(A),\mathtt{even}(B), n/2)$}
	\For {$i = 0 \dots n/2-1} $
		\State {$C[2i] \gets even[i]$}
		\State {$C[2i+1] \gets odd[i]$}
	\EndFor
	\For {$i = 0 \dots n/2-2} $
		\State {$\mathtt{Compare\mbox{-}Exchange}(C, 2i+1, 2i+2)$}
	\EndFor
\EndIf
\EndProcedure
\end{algorithmic}
\end{algorithm}

\subsection{The Main Algorithm}

Given the Odd-Even merge construction, we can perform a straight-forward merge sort.

The interesting part of Odd-Even Mergesort is perhaps that it is possible to construct a merging network that has so few prerequisites, which leads to an exceptionally simple algorithm. Unfortunately, the price of having a simple merge sort, is a more complex merge step.

The exact procedure for merge sorting using Odd-Even Merging is shown as pseudo-code in Algorithm~\ref{OddEvenMergeSort}

\begin{algorithm}
\caption{Odd-Even Mergesort}\label{OddEvenMergeSort}
\begin{algorithmic}[1]
	\Statex $A$: Array input of \texttt{Compare-Exchange} compatible elements
	\Statex $n$: Size of $A$
\Procedure{OddEvenMergeSort}{$A, n$}
\If {$n>1$}
	\State{$\mathtt{OddEvenMergeSort}(A[0 \dots n/2-1], n/2)$}
	\State{$\mathtt{OddEvenMergeSort}(A[n/2 \dots n-1], n/2)$}
	\State{$\mathtt{OddEvenMerge}(A[0 \dots n/2-1], A[n/2 \dots n-1], n)$}
\EndIf
\EndProcedure
\end{algorithmic}
\end{algorithm}

The running time of Odd-Even Mergesort, like Bitonic Sort follows from the Master Theorem  \textcolor{red}{and exercise 4.6-2} of~\citeB{Cormen}.
